\chapter{Results and Analyses} \label{chap:experiments}

\section*{}

In this chapter, some results of experiments done using the approach developed
(~\ref{chap:chap4}) and multiple datasets (with different characteristics).

There are also some experiments in order to understand which values are better
for some parameters and in which situations they work better or worst.

\section{Interval size without segmentation}

To test which interval of time gets better results for predicting time series
used on the approach, a dataset containing real data from 26-09-2013 to
27-01-2014 was used. The interval sizes tested were 4 hours, 6 hours, 8 hours,
12 hours and 24 hours.

For this test only the \emph{Volume Forecasting} phase, was used.

The time intervals used for the test were:
\begin{itemize}
\item case 1 - 26-09-2013 to 26-11-2013 for training;
27-11-2013 to 25-01-2014 for validation;
\item case 2 - 26-09-2013 to 26-12-2013 for training; 27-12-2013 to 25-01-2014
for validation;
\item case 3 - 26-09-2013 to 31-12-2013 for training; 01-01-2014 to 25-01-2014
for validation;
\end{itemize}

On tables ~\ref{tab:case1_interval}, ~\ref{tab:case2_interval} and
~\ref{tab:case3_interval}, for each one of the three cases the error of the
impressions volume forecasting for every test interval is shown.
In every case, the interval that got better results was the 12 hour interval
(minimum \emph{MASE}).

Additional errors values and graphs for this test are available on appendix
~\ref{ap:tests_result}

\begin{table}[!ht]
\footnotesize
\begin{tabular}{c|ccccccccc}
              & 4h baseline & 4h allow drift & 4h     &  & 6h baseline & 6h allow drift & 6h     &  &  \\ \hline
$\sigma$ (Real Data) & \multicolumn{3}{c}{946.25}           &  & \multicolumn{3}{c}{1358.24}           &  &  \\
RMSE          & 690.87      & 562.80         & 562.80 &  & 945.70      & 742.57         & 742.57 &  &  \\
MASE          & 0.5227      & 0.4397         & 0.4397 &  & 0.3659      & 0.3003         & 0.3003 &  & 
\end{tabular}

\vspace{0.5cm}

\begin{tabular}{c|ccccccccc}
              & 8h baseline & 8h allow drift & 8h     &  & 12h baseline & \color{red}{12h allow drift} & 12h     &  &  \\ \hline
$\sigma$ (Real Data) & \multicolumn{3}{c}{1785.84}           &  & \multicolumn{3}{c}{2224.74}           &  &  \\
RMSE          & 1150.43     & 955.35         & 955.35 &  & 1555.01     & \color{red}{1233.82}     & 1233.82 &  &  \\
MASE          & 0.2712      & 0.2473         & 0.2473 &  & 0.2253      & \color{red}{0.1967}      & 0.1967 &  & 
\end{tabular}

\vspace{0.5cm}

\begin{tabular}{c|ccc}
              & 24h baseline & 24h allow drift & 24h  \\ \hline
$\sigma$ (Real Data) & \multicolumn{3}{c}{1946.70}     \\
RMSE          & 2448.97     & 2139.99        & 2139.99   \\
MASE          & 1.4338      & 1.2725         & 1.2725 \\
\end{tabular}

\vspace{0.5cm}

\caption{Case 1: Forecast errors for different interval sizes (best result in
red)}\label{tab:case1_interval}
\end{table}




\begin{table}[!ht]
\footnotesize
\begin{tabular}{c|ccccccccc}
              & 4h baseline & 4h allow drift & 4h     &  & 6h baseline & 6h allow drift & 6h     &  &  \\ \hline
$\sigma$ (Real Data) & \multicolumn{3}{c}{908.70}           &  & \multicolumn{3}{c}{1293.78}           &  &  \\
RMSE          & 727.31      & 1083.22        & 834.32 &  & 967.46      & 740.73         & 740.73 &  &  \\
MASE          & 0.1922      & 0.3152         & 0.2348 &  & 0.1312      & 0.0947         & 0.0947 &  & 
\end{tabular}

\vspace{0.5cm}

\begin{tabular}{c|ccccccccc}
              & 8h baseline & 8h allow drift & 8h     &  & \color{red}{12h baseline} & 12h
              allow drift & 12h     &  &  \\ \hline
$\sigma$ (Real Data) & \multicolumn{3}{c}{1691.32}           &  & \multicolumn{3}{c}{2154.85}           &  &  \\
RMSE          & 1205.49      & 999.47        & 999.47 &  & \color{red}{1565.43}    & 1722.83        & 1722.83 &  &  \\
MASE          & 0.0983      & 0.0905         & 0.0905 &  & \color{red}{0.0857}     & 0.0953         & 0.0953 &  & 
\end{tabular}

\vspace{0.5cm}

\begin{tabular}{c|ccc}
              & 24h baseline & 24h allow drift & 24h  \\ \hline
$\sigma$ (Real Data) & \multicolumn{3}{c}{1752.65}     \\
RMSE          & 2329.76     & 1646.92         & 1646.92   \\
MASE          & 0.4198      & 0.2925         & 0.2925 \\
\end{tabular}

\vspace{0.5cm}

\caption{Case 2: Forecast errors for different interval sizes (best result in
red)}\label{tab:case2_interval}
\end{table}


\begin{table}[!ht]
\footnotesize
\begin{tabular}{c|ccccccccc}
              & 4h baseline & 4h allow drift & 4h     &  & 6h baseline & 6h allow drift & 6h     &  &  \\ \hline
$\sigma$ (Real Data) & \multicolumn{3}{c}{929.73}           &  & \multicolumn{3}{c}{1334.01}           &  &  \\
RMSE          & 737.22      & 692.61         & 692.61 &  & 985.86      & 930.13        & 930.13 &  &  \\
MASE          & 0.1548      & 0.1441         & 0.1441 &  & 0.1071      & 0.0998        & 0.0998 &  & 
\end{tabular}

\vspace{0.5cm}

\begin{tabular}{c|ccccccccc}
              & 8h baseline & 8h allow drift & 8h     &  & 12h baseline & \color{red}{12h allow drift} & 12h     &  &  \\ \hline
$\sigma$ (Real Data) & \multicolumn{3}{c}{1738.75}           &  & \multicolumn{3}{c}{2203.12}           &  &  \\
RMSE          & 1197.45     & 1165.13        & 1165.13  &  & 1581.66     & \color{red}{1453.84}      & 1453.84 &  &  \\
MASE          & 0.0778      & 0.0763         & 0.0763  &  & 0.0695      & \color{red}{0.0595}        & 0.0595 &  & 
\end{tabular}

\vspace{0.5cm}

\begin{tabular}{c|ccc}
              & 24h baseline & 24h allow drift & 24h  \\ \hline
$\sigma$ (Real Data) & \multicolumn{3}{c}{1686.92}     \\
RMSE          & 2275.91     & 2547.73         & 2260.99   \\
MASE          & 0.3304      & 0.3525         & 0.3044 \\
\end{tabular}

\vspace{0.5cm}

\caption{Case 3: Forecast errors for different interval sizes (best result in
red)}\label{tab:case3_interval}
\end{table}


\section{Segmentation}

The experiments described on this section use all the phases of the approach.
This experiments will use datasets generated on artificial environment in order to
test the ability of the approach to capture changes on the volumes of some
characteristics of the dataset.
It will also be used a real world dataset, in order to assess the result in a
more realist conditions.

\subsection{Particular browser increasing}

\subsection*{Without segmentation}

For the first test it was used an artificial generated dataset containing a constant overall
volume of impressions, but with the particularity of an increasing volume of
impressions from users using ''Safari 4.0'' browser. 

\begin{figure}[!ht]
\centering
\begin{minipage}[t]{0.45\linewidth}
\includegraphics[width=0.96\textwidth]{forecast_12_wo_clustering} \caption[Volume
impression forecast, safari]{Volume impression
forecast, using 12h period without clustering (blue: real; green: forecast)}
\label{fig:vol_safari_12h_wo_clustering}
\end{minipage}
\quad
\begin{minipage}[t]{0.45\linewidth}
\includegraphics[width=0.96\textwidth]{safari4_wo_clustering_12} \caption[Volume
impression forecast, safari 4]{Volume impression
forecast, using 12h period without clustering, filtered by "Safari 4.0" (blue: real; green: forecast)}
\label{fig:vol_safari_12h_wo_clustering_safari_4} 
\end{minipage}
\end{figure}



\begin{table}[!ht]
\centering
\footnotesize
\begin{minipage}[t]{0.45\linewidth}
\centering
\begin{tabular}{ccc}
 $\sigma$ (Real Data) & RMSE & MASE   \\ \hline
 3.33      & 19.61        & 0.6760   \\
\end{tabular}

\vspace{0.5cm}

\caption[Volume
impression forecast, safari]{Error for impression volume
forecast, using a 12h period without clustering}
\label{tab:err_forecast_12_safari}
\end{minipage}
\quad
\begin{minipage}[t]{0.45\linewidth}
\centering
\begin{tabular}{ccc}
 $\sigma$ (Real Data) & RMSE & MASE   \\ \hline
72.28      &  155.42       & 19.5770   \\
\end{tabular}

\vspace{0.5cm}

\caption[Volume
impression forecast, safari]{Error for impression volume
forecast, using a 12h period without clustering, query for ''Safari 4.0''}
\label{tab:err_forecast_12_safari_wo_clustering_safari_4}

\end{minipage}
\end{table}

As shown in the figure~\ref{fig:vol_safari_12h_wo_clustering} and in the
table~\ref{tab:err_forecast_12_safari}, (without the segmentation phase) the
proposed approach was able to forecast the total volume of impressions only with
a small error.

In order to be able to see the result of the approach for the behavior of the
''Safari 4.0'' users we need to generate a future dataset (last phase of the
approach). The result for the query for the ''Safari 4.0'' browser, over the
resultant dataset, can be seen
on the figure~\ref{fig:vol_safari_12h_wo_clustering_safari_4}.

As shown on the figure~\ref{fig:vol_safari_12h_wo_clustering_safari_4}, the
dataset generation phased mimic the behavior of that particular characteristic
from the previous week into the future, which in this case is not a good result.

\subsection*{Clustering Baseline}

In order to better compare the results of the different segmentation approaches
a baseline method was used\footnote{the original dataset was grouped in a
chronological order on 100 different groups.}.

\begin{figure}[!ht]
\centering
\begin{minipage}[t]{0.45\linewidth}
\includegraphics[width=0.96\textwidth]{forecast_12_w_clustering_baseline} \caption[Volume
impression forecast, safari 4]{Volume impression
forecast, using 12h period with baseline clustering  (blue: real; green: forecast)}
\label{fig:vol_safari_12h_w_clustering_baseline}
\end{minipage}
\quad
\begin{minipage}[t]{0.45\linewidth}
\includegraphics[width=0.96\textwidth]{forecast_12_w_clustering_baseline_safari4} \caption[Volume
impression forecast, safari 4]{Volume impression
forecast, using 12h period with baseline clustering, filtered by "Safari 4.0" (blue: real; green: forecast)}
\label{fig:vol_safari_12h_w_clustering_baseline_safari_4}
\end{minipage}

\end{figure}

\begin{table}[!ht]
\centering
\footnotesize
\begin{minipage}[t]{0.45\linewidth}
\centering
\footnotesize
\begin{tabular}{ccc}
 $\sigma$ (Real Data) & RMSE & MASE   \\ \hline
3.33 & 3.38 & 0.0931 \\
\end{tabular}

\vspace{0.5cm}

\caption[Volume
impression forecast, safari]{Error for impression volume
forecast, using 12h period with baseline clustering}
\label{tab:err_forecast_12_safari_w_clustering_datastream_14}
\end{minipage}
\quad
\begin{minipage}[t]{0.45\linewidth}
\centering
\footnotesize
\begin{tabular}{ccc}
 $\sigma$ (Real Data) & RMSE & MASE   \\ \hline
72.28 & 154.99 & 19.5144 \\
\end{tabular}

\vspace{0.5cm}

\caption[Volume
impression forecast, safari]{Error for impression volume
forecast, using 12h period with baseline clustering, filtered by "Safari 4.0"}
\label{tab:err_forecast_12_safari_w_clustering_datastream_14}
\end{minipage}

\end{table}

On figure~\ref{fig:vol_safari_12h_w_clustering_baseline} and table~\ref{tab:err_forecast_12_safari_w_clustering_datastream_14}
we can assess that this approach gets a slightly improvement over the
non-segmented approach. If we look at the filtered data
(figure~\ref{fig:vol_safari_12h_w_clustering_baseline_safari_4}) the improvement
is not that noticeable.


\subsection*{Segmentation by Parameter}

\subsubsection*{Segmentation by browser}

Since we know that this particular dataset has a peculiarity in the browser
parameter impression volume. We test how the prediction would react to the clustering by
parameter, in this case by the browser attribute, this will forecast the volumes
values for each browser present on the dataset allowing to get better results in
predicting the traffic volumes for each browser.

As we can see on picture~\ref{fig:vol_safari_12h_w_clustering}, the overall error
(table~\ref{tab:err_forecast_12_safari_w_clustering}) for impression volume
forecasting is worst than without the clustering phase this is due to the error
for prediction associated with each cluster is amplified when the volume data is joined
back together.

\begin{figure}[!ht]
\centering
\begin{minipage}[b]{0.45\linewidth}
\includegraphics[width=0.96\textwidth]{forecast_12_w_clustering_per_browser} \caption[Volume
impression forecast, safari 4]{Volume impression
forecast, using 12h period with clustering by the browser attribute (blue: real; green: forecast)}
\label{fig:vol_safari_12h_w_clustering}
\end{minipage}
\quad
\begin{minipage}[b]{0.45\linewidth}
\includegraphics[width=0.96\textwidth]{safari4_w_clustering_12} \caption[Volume
impression forecast, safari 4]{Volume impression
forecast, using 12h period with clustering by the browser parameter, filtered by "Safari 4.0" (blue: real; green: forecast)}
\label{fig:vol_safari_12h_w_clustering_safari_4}
\end{minipage}

\end{figure}

\begin{table}[!ht]
\centering
\footnotesize
\begin{minipage}[t]{0.45\linewidth}
\centering

\footnotesize
\begin{tabular}{ccc}
 $\sigma$ (Real Data) & RMSE & MASE   \\ \hline
3.33 & 136.78 & 4.4253 \\
\end{tabular}

\vspace{0.5cm}

\caption[Volume
impression forecast, safari]{Error for impression volume
forecast, using a 12h period with clustering by browser attribute}
\label{tab:err_forecast_12_safari_w_clustering}
\end{minipage}
\quad
\begin{minipage}[t]{0.45\linewidth}
\centering
\footnotesize
\begin{tabular}{ccc}
 $\sigma$ (Real Data) & RMSE & MASE   \\ \hline
72.28 & 2.58 & 0.2892 \\
\end{tabular}

\vspace{0.5cm}

\caption[Volume
impression forecast, safari]{Error for impression volume
forecast, using a 12h period with clustering by browser attribute, query for ''Safari 4.0''}
\label{tab:err_forecast_12_safari_w_clustering_safari_4}

\end{minipage}

\end{table}

In the other hand, in the
figure~\ref{fig:vol_safari_12h_w_clustering_safari_4}, which show the results
filtered by ''Safari 4.0'' browser, we get a better result prediction for this
characteristic, as shown by the error values on
table~\ref{tab:err_forecast_12_safari_w_clustering_safari_4}. So the
\emph{clustering by parameter} allow to get better prediction for certain
characteristics by capturing the volume for each one in the past and propagate
it into the future.

\subsection*{Using datastream-based clustering}

\begin{figure}[!ht]
\centering
\begin{minipage}[t]{0.45\linewidth}
\includegraphics[width=0.96\textwidth]{forecast_12_w_clustering_datastream_14} \caption[Volume
impression forecast, safari 4]{Volume impression
forecast, using 12h period with clustering based datastream-based clustering,
with a \emph{threshold} of 14 maximum distance  (blue: real; green: forecast)}
\label{fig:vol_safari_12h_w_clustering_datastream_14}
\end{minipage}
\quad
\begin{minipage}[t]{0.45\linewidth}
\includegraphics[width=0.96\textwidth]{safari4_w_clustering_12_datastream14} \caption[Volume
impression forecast, safari 4]{Volume impression
forecast, using 12h period with clustering based datastream-based clustering,
with a \emph{threshold} of 14 maximum distance, filtered by "Safari 4.0" (blue: real; green: forecast)}
\label{fig:vol_safari_12h_w_clustering_datastream_14_safari_4}
\end{minipage}

\end{figure}

\begin{table}[!ht]
\centering
\footnotesize
\begin{minipage}[t]{0.45\linewidth}
\centering
\footnotesize
\begin{tabular}{ccc}
 $\sigma$ (Real Data) & RMSE & MASE   \\ \hline
3.33 & 50.41 & 1.6226 \\
\end{tabular}

\vspace{0.5cm}

\caption[Volume
impression forecast, safari]{Error for impression volume
forecast, using 12h period with clustering based datastream-based clustering,
with a \emph{threshold} of 14 maximum distance }
\label{tab:err_forecast_12_safari_w_clustering_datastream_14}
\end{minipage}
\quad
\begin{minipage}[t]{0.45\linewidth}
\centering
\footnotesize
\begin{tabular}{ccc}
 $\sigma$ (Real Data) & RMSE & MASE   \\ \hline
72.28 & 84.78 & 10.5275 \\
\end{tabular}

\vspace{0.5cm}

\caption[Volume
impression forecast, safari]{Error for impression volume
forecast, using 12h period with clustering based datastream-based clustering,
with a \emph{threshold} of 14 maximum distance, filtered by "Safari 4.0"}
\label{tab:err_forecast_12_safari_w_clustering_datastream_14}
\end{minipage}

\end{table}

Using the datastream-based clustering method, with a \emph{threshold} of 14
(which means that each member of each groups will only what a maximum of 14
parameters different than the centroid of the group) the result for the overall
impressions volume was worst than without clustering and the baseline clustering
method, but better than the clustering by browser. If we analyse the
figure~\ref{fig:vol_safari_12h_w_clustering_datastream_14_safari_4}, we can
assess than the result is only beat by the clustering for that specific
parameter, which ultimately is a good result.

\subsection{Specific Domain decreasing linearly}

For this set of tests another artificially generated dataset was used, in this
particular case one of the domains represented on the dataset, decreases its
volume following an linear function.

\subsection*{Without Segmentation}

\begin{figure}[!ht]
\centering
\begin{minipage}[t]{0.45\linewidth}
\includegraphics[width=0.96\textwidth]{results_3/wo_clustering} \caption[Volume
impression forecast, without segmentation]{Impression volume
forecast, using 12h period without segmentation (blue: real; green: forecast)}
\label{fig:vol_domain_wo_segmentation}
\end{minipage}
\quad
\begin{minipage}[t]{0.45\linewidth}
\includegraphics[width=0.96\textwidth]{results_3/host_wo_clustering} \caption[Volume
impression forecast, without segmentation]{Impression volume
forecast, using 12h period without segmentation, filtered by the domain ''ef4e08fb71d96d19406663f8bb7ce6c0'' (blue: real; green: forecast)}
\label{fig:vol_domain_wo_segmentation_filtered}
\end{minipage}

\end{figure}

\begin{table}[!ht]
\centering
\footnotesize
\begin{minipage}[t]{0.45\linewidth}
\centering
\footnotesize
\begin{tabular}{ccc}
 $\sigma$ (Real Data) & RMSE & MASE   \\ \hline
4.33 & 5.00 & 0.1392 \\
\end{tabular}

\vspace{0.5cm}

\caption[Volume
impression forecast, safari]{Error for impression volume
forecast, using 12h period without segmentation }
\label{tab:err_domain_wo_segmentation}
\end{minipage}
\quad
\begin{minipage}[t]{0.45\linewidth}
\centering
\footnotesize
\begin{tabular}{ccc}
 $\sigma$ (Real Data) & RMSE & MASE   \\ \hline
72.50 & 152.78 & 12.8542 \\
\end{tabular}

\vspace{0.5cm}

\caption[Volume
impression forecast, safari]{Error for impression volume
forecast, using 12h period without segmentation, filtered by the domain ''ef4e08fb71d96d19406663f8bb7ce6c0'' }
\label{tab:err_domain_wo_segmentation_filtered}
\end{minipage}

\end{table}

Without any segmentation phase we obtained a good result for the overall
impression volume prediction. If we filter the results for the particular
domain,
that we know will be decreasing, the result mimics the behaviour of the previous
weeks since the information that it was about that particular domain is limited.

\subsection*{Clustering Baseline}

\begin{figure}[!ht]
\centering
\begin{minipage}[t]{0.45\linewidth}
\includegraphics[width=0.96\textwidth]{results_3/baseline} \caption[Volume
impression forecast, domain, cluster by baseline]{Impression volume
forecast, using 12h period using baseline segmentation  (blue: real; green: forecast)}
\label{fig:domain_w_baseline}
\end{minipage}
\quad
\begin{minipage}[t]{0.45\linewidth}
\includegraphics[width=0.96\textwidth]{results_3/host_baseline} \caption[Volume
impression forecast, domain, cluster by baseline, filtered]{Impression volume
forecast, using 12h period using baseline segmentation, filtered by the domain ''ef4e08fb71d96d19406663f8bb7ce6c0'' (blue: real; green: forecast)}
\label{fig:domain_w_baseline_filtered}
\end{minipage}

\end{figure}

\begin{table}[!ht]
\centering
\footnotesize
\begin{minipage}[t]{0.45\linewidth}
\centering
\footnotesize
\begin{tabular}{ccc}
 $\sigma$ (Real Data) & RMSE & MASE   \\ \hline
4.33 & 4.30 & 0.1176 \\
\end{tabular}

\vspace{0.5cm}

\caption[Error Volume
impression forecast, domain, filtered]{Error for impression volume
forecast, using 12h period with baseline segmentation}
\label{tab:err_domain_w_segmentation_baseline}
\end{minipage}
\quad
\begin{minipage}[t]{0.45\linewidth}
\centering
\footnotesize
\begin{tabular}{ccc}
 $\sigma$ (Real Data) & RMSE & MASE   \\ \hline
72.50 & 150.7442 & 12.6627 \\
\end{tabular}

\vspace{0.5cm}

\caption[Error Volume
impression forecast, domain, filtered]{Error for impression volume
forecast, using 12h period with baseline segmentation, filtered by the domain ''ef4e08fb71d96d19406663f8bb7ce6c0'' }
\label{tab:err_domain_w_segmentation_baseline_filtered}
\end{minipage}

\end{table}


In the figure~\ref{fig:domain_w_baseline} and
table~\ref{tab:err_domain_w_segmentation_baseline} we can take a look at the
results given by the segmentation baseline method for the total volume of
impressions forecast for this dataset, as we can assess the results are better
than without segmentation, and when we filter the data for the domain that is
decreasing in volume of impressions (figure~\ref{fig:domain_w_baseline_filtered}
and table~\ref{tab:err_domain_w_segmentation_baseline_filtered}) we also obtain
a slightly better result.

\subsection*{Segmentation by Parameter}

\subsubsection*{Segmentation by Domain}

\begin{figure}[!ht]
\centering
\begin{minipage}[t]{0.45\linewidth}
\includegraphics[width=0.96\textwidth]{results_3/per_host}

\caption[Volume
impression forecast, domain, cluster by domain]{Impression volume
forecast, using 12h period using segmentation by domain (blue: real; green: forecast)}
\label{fig:domain_w_domain}


\end{minipage}
\quad
\begin{minipage}[t]{0.45\linewidth}
\includegraphics[width=0.96\textwidth]{results_3/host_per_host} 
\caption[Volume
impression forecast, domain, cluster by domain, filtered]{Impression volume
forecast, using 12h period using segmentation by domain, filtered by the domain ''ef4e08fb71d96d19406663f8bb7ce6c0'' (blue: real; green: forecast)}
\label{fig:domain_w_domain_filtered}

\end{minipage}

\end{figure}

\begin{table}[!ht]
\centering
\footnotesize
\begin{minipage}[t]{0.45\linewidth}
\centering
\footnotesize
\begin{tabular}{ccc}
 $\sigma$ (Real Data) & RMSE & MASE   \\ \hline
4.33 & 283.20 & 9.9304 \\
\end{tabular}

\vspace{0.5cm}

\caption[Error Volume
impression forecast, domain]{Error for impression volume
forecast, using 12h period with segmentation by domain}
\label{tab:err_domain_w_segmentation_domain}


\end{minipage}
\quad
\begin{minipage}[t]{0.45\linewidth}
\centering
\footnotesize
\begin{tabular}{ccc}
 $\sigma$ (Real Data) & RMSE & MASE   \\ \hline
72.50 & 137.93 & 11.2913 \\
\end{tabular}

\vspace{0.5cm}

\caption[Error Volume
impression forecast, domain, filtered]{Error for impression volume
forecast, using 12h period with segmentation by domain, filtered by the domain ''ef4e08fb71d96d19406663f8bb7ce6c0'' }
\label{tab:err_domain_w_segmentation_domain_filtered}


\end{minipage}

\end{table}

Like we did for the previous test group where we knew that a particular browser
was increasing its value in volume impressions we segmented the data by browser,
in this case we know that a domain is decreasing, so we tested segment the data
by domain. In the figure~\ref{fig:domain_w_domain_filtered} the results where
not as expected, although we got a better result in the filtered data, when
compared to the previous methods, the result for the total volume of that is
much worse. This is probably due to small volumes in each domain since in this
dataset was about 4300 domains in about 3000 impressions in each 12h period
which can result in a very difficult to predict time series.


\subsection*{Using datastream-based clustering}

\begin{figure}[!ht]
\centering
\begin{minipage}[t]{0.45\linewidth}
\includegraphics[width=0.96\textwidth]{results_3/datastream_14} 
\caption[Volume
impression forecast, domain, cluster by datastream]{Impression volume
forecast, using 12h period using datastream-based clustering with a
\emph{threshold} of 14 (blue: real; green: forecast)}
\label{fig:domain_w_datastream}

\end{minipage}
\quad
\begin{minipage}[t]{0.45\linewidth}
\includegraphics[width=0.96\textwidth]{results_3/host_datastream_14.png} 
\caption[Volume
impression forecast, domain, cluster by datastream, filtered]{Impression volume
forecast, using 12h period using datastream-based clustering with a
\emph{threshold} of 14, filtered by the domain ''ef4e08fb71d96d19406663f8bb7ce6c0'' (blue: real; green: forecast)}
\label{fig:domain_w_datastream_filtered}

\end{minipage}

\end{figure}

\begin{table}[!ht]
\centering
\footnotesize
\begin{minipage}[t]{0.45\linewidth}
\centering
\footnotesize
\begin{tabular}{ccc}
 $\sigma$ (Real Data) & RMSE & MASE   \\ \hline
4.33 & 32.48 & 1.0613 \\
\end{tabular}

\vspace{0.5cm}

\caption[Error Volume
impression forecast, datastream, filtered]{Error for impression volume
forecast, using 12h period using datastream-based clustering with a
\emph{threshold} of 14}
\label{tab:err_domain_w_datastream}


\end{minipage}
\quad
\begin{minipage}[t]{0.45\linewidth}
\centering
\footnotesize
\begin{tabular}{ccc}
 $\sigma$ (Real Data) & RMSE & MASE   \\ \hline
72.50 & 123.27 & 10.3919 \\
\end{tabular}

\vspace{0.5cm}

\caption[Error Volume
impression forecast, domain, filtered]{Error for impression volume
forecast, using 12h period using datastream-based clustering with a
\emph{threshold} of 14, filtered by the domain ''ef4e08fb71d96d19406663f8bb7ce6c0'' }
\label{tab:err_domain_w_segmentation_datastream_filtered}


\end{minipage}

\end{table}

The datastream-based segmentation approach using a \emph{threshold} 14 of
maximum distance, we get a slightly worse result than without segmentation and
the baseline segmentation approach, as shown by
figure~\ref{fig:domain_w_datastream} and
table~\ref{tab:err_domain_w_datastream}. Is by using this method that we got the
better results on this series, in terms of capturing the peculiarity of this
dataset, as shown by figure~\ref{fig:domain_w_datastream_filtered} and
table~\ref{tab:err_domain_w_segmentation_datastream_filtered}.

\subsection{Real Data}

For this last series of tests we used a dataset containing real data, since we
don't know any particular characteristic to search for we will compare the
results for the query for ''Portugal'' as country of origin of each impression.

\subsection*{Without Segmentation}

\begin{figure}[!ht]
\centering
\begin{minipage}[t]{0.45\linewidth}
\includegraphics[width=0.96\textwidth]{results_real/wo_clustering}
\caption[Volume
impression forecast, real data]{Impression volume
forecast, using 12h period without clustering (blue: real; green: forecast)}
\label{fig:vol_real_data_wo_clustering}
\end{minipage}
\quad
\begin{minipage}[t]{0.45\linewidth}
\includegraphics[width=0.96\textwidth]{results_real/wo_clustering_portugal} \caption[Volume
impression forecast, real data, Portugal]{Impression volume
forecast, using 12h period without clustering, filtered by country = Portugal (blue: real; green: forecast)}
\label{fig:vol_real_data_wo_clustering_filtered}
\end{minipage}

\end{figure}

\begin{table}[!ht]
\centering
\footnotesize
\begin{minipage}[t]{0.45\linewidth}
\centering
\footnotesize
\begin{tabular}{ccc}
 $\sigma$ (Real Data) & RMSE & MASE   \\ \hline
2224.74 & 1237.99 & 0.1958 \\
\end{tabular}

\vspace{0.5cm}

\caption[Volume
impression forecast, real data, without clustering]{Error for impression volume
forecast, using 12h period without clustering}
\label{tab:err_real_data_wo_clustering}
\end{minipage}
\quad
\begin{minipage}[t]{0.45\linewidth}
\centering
\footnotesize
\begin{tabular}{ccc}
 $\sigma$ (Real Data) & RMSE & MASE   \\ \hline
6.70 & 9.24 & 1.2511 \\
\end{tabular}

\vspace{0.5cm}

\caption[Volume
impression forecast, safari]{Error for impression volume
forecast, using 12h period without clustering, filtered by country = Portugal}
\label{tab:err_real_data_wo_clustering_filtered}
\end{minipage}

\end{table}

Like in the other datasets we start by testing the result without any
segmentation. In overall volume of impressions we get a good result
(figure~\ref{fig:vol_real_data_wo_clustering} and
table~\ref{tab:err_real_data_wo_clustering}), and when over the result a query
for origin country ''Portugal'' is done, the result mimics the past
characteristics and the result on this case is not bad.


\subsection*{Clustering Baseline}

\begin{figure}[!ht]
\centering
\begin{minipage}[t]{0.45\linewidth}
\includegraphics[width=0.96\textwidth]{results_real/baseline} \caption[Volume
impression forecast, real data, clustering baseline]{Impression Volume 
forecast, using 12h period with baseline segmentation (blue: real; green: forecast)}
\label{fig:vol_real_data_baseline}
\end{minipage}
\quad
\begin{minipage}[t]{0.45\linewidth}
\includegraphics[width=0.96\textwidth]{results_real/baseline_portugal} \caption[Volume
impression forecast, real data, clustering baselinei, filtered]{Impression Volume 
forecast, using 12h period with baseline segmentation, filtered by country =
Portugal (blue: real; green: forecast)}
\label{fig:vol_real_data_baseline_filtered}
\end{minipage}

\end{figure}

\begin{table}[!ht]
\centering
\footnotesize
\begin{minipage}[t]{0.45\linewidth}
\centering
\footnotesize
\begin{tabular}{ccc}
 $\sigma$ (Real Data) & RMSE & MASE   \\ \hline
2224.74 & 1352.42 & 0.2174 \\
\end{tabular}

\vspace{0.5cm}

\caption[Volume
impression forecast, baseline]{Error for volume impression forecast, using 12h
period with baseline clustering}
\label{tab:err_forecast_12_real_data_baseline}
\end{minipage}
\quad
\begin{minipage}[t]{0.45\linewidth}
\centering
\footnotesize
\begin{tabular}{ccc}
 $\sigma$ (Real Data) & RMSE & MASE   \\ \hline
6.70 & 6.88 & 0.7896 \\
\end{tabular}

\vspace{0.5cm}

\caption[Volume
impression forecast, safari]{Error for impression volume
forecast, using 12h period with baseline clustering, filtered by country =
Portugal}
\label{tab:err_forecast_12_real_data_baseline_filtered}
\end{minipage}

\end{table}

The baseline segmentation, in this case gets a worse result than the
non-segmented approach in terms of total volume, but a better approximation
in terms of the results for the query ''Portugal'', probably due to additional
granularity gained by the data division.


\subsection*{Segmentation by Parameter}

\subsubsection*{Segmentation by browser}

\begin{figure}[!ht]
\centering
\begin{minipage}[t]{0.45\linewidth}
\includegraphics[width=0.96\textwidth]{results_real/perbrowser} \caption[Volume
impression forecast, real data, clustering by browser]{Impression Volume 
forecast, using 12h period with segmentation by browser(blue: real; green: forecast)}
\label{fig:vol_real_data_browser}
\end{minipage}
\quad
\begin{minipage}[t]{0.45\linewidth}
\includegraphics[width=0.96\textwidth]{results_real/perbrowser_portugal}\caption[Volume
impression forecast, real data, clustering by browser, filtered]{Impression Volume 
forecast, using 12h period with segmentation by browser, filtered by country =
Portugal (blue: real; green: forecast)}
\label{fig:vol_real_data_browser_filtered}
\end{minipage}

\end{figure}

\begin{table}[!ht]
\centering
\footnotesize
\begin{minipage}[t]{0.45\linewidth}
\centering
\footnotesize
\begin{tabular}{ccc}
 $\sigma$ (Real Data) & RMSE & MASE   \\ \hline
2224.74 & 1226.60 & 0.1884 \\
\end{tabular}

\vspace{0.5cm}

\caption[Volume
impression forecast, real data, browser]{Error for impression volume
forecast, using 12h period with segmentation by browser}
\label{tab:err_forecast_12_real_data_browser}
\end{minipage}
\quad
\begin{minipage}[t]{0.45\linewidth}
\centering
\footnotesize
\begin{tabular}{ccc}
 $\sigma$ (Real Data) & RMSE & MASE   \\ \hline
6.70 & 10.75 & 1.3883 \\
\end{tabular}

\vspace{0.5cm}

\caption[Volume
impression forecast, real data, browser, filtered]{Error for impression volume
forecast, using 12h period with segmentation by browser, filtered by country =
Portugal}
\label{tab:err_forecast_12_real_data_browser_filtered}
\end{minipage}

\end{table}

Since, in the case of this dataset we do not know for which particular parameter
we might want to segment by, the browser parameter was randomly selected as the
segmentation parameter. For the overall volume result this was the best result
over the other approaches over this particular dataset. In terms of the query
for ''Portugal'' the result was slightly worse than previous approaches.

\subsection*{Using datastream-based clustering}

\begin{figure}[!ht]
\centering
\begin{minipage}[t]{0.45\linewidth}
\includegraphics[width=0.96\textwidth]{results_real/datastream_20} \caption[Volume
impression forecast, real data, clustering datastream]{Impression Volume 
forecast, using 12h period with datastream-based segmentation using
\emph{threshold} of 20 (blue: real; green: forecast)}
\label{fig:vol_real_data_datastream}
\end{minipage}
\quad
\begin{minipage}[t]{0.45\linewidth}
\includegraphics[width=0.96\textwidth]{results_real/datastream_20_portugal} \caption[Volume
impression forecast, real data, clustering datastream, filtered]{Volume impression
forecast, using 12h period with datastream-based segmentation using
\emph{threshold} of 20, filtered by country = Portugal  (blue: real; green: forecast)}
\label{fig:vol_real_data_datastream_filtered}
\end{minipage}

\end{figure}

\begin{table}[!ht]
\centering
\footnotesize
\begin{minipage}[t]{0.45\linewidth}
\centering
\footnotesize
\begin{tabular}{ccc}
 $\sigma$ (Real Data) & RMSE & MASE   \\ \hline
2224.74 & 1786.43 & 0.2927 \\
\end{tabular}

\vspace{0.5cm}

\caption[Volume
impression forecast, real data, datastream]{Error for impression volume
forecast, using 12h period with datastream-based segmentation using
\emph{threshold} of 20}
\label{tab:err_forecast_12_real_datastream_filtered}
\end{minipage}
\quad
\begin{minipage}[t]{0.45\linewidth}
\centering
\footnotesize
\begin{tabular}{ccc}
 $\sigma$ (Real Data) & RMSE & MASE   \\ \hline
6.70 & 15.93 & 2.5960 \\
\end{tabular}

\vspace{0.5cm}

\caption[Volume
impression forecast, real data, datastream, filtered]{Error for impression volume
forecast, using 12h period with datastream-based segmentation using
\emph{threshold} of 20, filtered by country = Portugal}
\label{tab:err_forecast_12_real_datastream_filtered}
\end{minipage}

\end{table}

For this dataset the \emph{threshold} for the distance needed to be increased to
20 (out of 28) because with a smaller \emph{threshold} we would get to many
groups, and some of them with data in the future not represented on the training
dataset.

In overall analysis this was the worst result for this dataset for impression
volume forecast and for the queried data. This bad result can be associated with
the associated results did not have any particular characteristic in terms of
volume trend that made sense together, but were group because they had similar
parameters.

\section{Conclusion}

