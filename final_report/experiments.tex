\chapter{Results and Analyses} \label{chap:experiments}

\section*{}

In this chapter, some results of experiments done using the approach developed
(~\ref{chap:chap4}) and multiple datasets (with different characteristics).

There are also some experiments in order to understand which values are better
for some parameters and in which situations they work better or worst.

\section{Interval size without segmentation}

To test which interval of time gets better results for predicting time series
used on the approach, a dataset containing real data from 26-09-2013 to
27-01-2014 was used. The interval sizes tested were 4 hours, 6 hours, 8 hours,
12 hours and 24 hours.

For this test only the \emph{Volume Forecasting} phase, was used.

The time intervals used for the test were:
\begin{itemize}
\item case 1 - 26-09-2013 to 26-11-2013 for training;
27-11-2013 to 25-01-2014 for validation;
\item case 2 - 26-09-2013 to 26-12-2013 for training; 27-12-2013 to 25-01-2014
for validation;
\item case 3 - 26-09-2013 to 31-12-2013 for training; 01-01-2014 to 25-01-2014
for validation;
\end{itemize}

On tables ~\ref{tab:case1_interval}, ~\ref{tab:case2_interval} and
~\ref{tab:case3_interval}, for each one of the three cases the error of the
impressions volume forecasting for every test interval is shown.
In every case, the interval that got better results was the 12 hour interval
(minimum \emph{MASE}).

Additional errors values and graphs for this test are available on appendix
~\ref{ap:tests_result}

\begin{table}[!ht]
\footnotesize
\begin{tabular}{c|ccccccccc}
              & 4h baseline & 4h allow drift & 4h     &  & 6h baseline & 6h allow drift & 6h     &  &  \\ \hline
$\sigma$ (Real Data) & \multicolumn{3}{c}{946.25}           &  & \multicolumn{3}{c}{1358.24}           &  &  \\
RMSE          & 690.87      & 562.80         & 562.80 &  & 945.70      & 742.57         & 742.57 &  &  \\
MASE          & 0.5227      & 0.4397         & 0.4397 &  & 0.3659      & 0.3003         & 0.3003 &  & 
\end{tabular}

\vspace{0.5cm}

\begin{tabular}{c|ccccccccc}
              & 8h baseline & 8h allow drift & 8h     &  & 12h baseline & \color{red}{12h allow drift} & 12h     &  &  \\ \hline
$\sigma$ (Real Data) & \multicolumn{3}{c}{1785.84}           &  & \multicolumn{3}{c}{2224.74}           &  &  \\
RMSE          & 1150.43     & 955.35         & 955.35 &  & 1555.01     & \color{red}{1233.82}     & 1233.82 &  &  \\
MASE          & 0.2712      & 0.2473         & 0.2473 &  & 0.2253      & \color{red}{0.1967}      & 0.1967 &  & 
\end{tabular}

\vspace{0.5cm}

\begin{tabular}{c|ccc}
              & 24h baseline & 24h allow drift & 24h  \\ \hline
$\sigma$ (Real Data) & \multicolumn{3}{c}{1946.70}     \\
RMSE          & 2448.97     & 2139.99        & 2139.99   \\
MASE          & 1.4338      & 1.2725         & 1.2725 \\
\end{tabular}

\vspace{0.5cm}

\caption{Case 1: Forecast errors for different interval sizes (best result in
red)}\label{tab:case1_interval}
\end{table}




\begin{table}[!ht]
\footnotesize
\begin{tabular}{c|ccccccccc}
              & 4h baseline & 4h allow drift & 4h     &  & 6h baseline & 6h allow drift & 6h     &  &  \\ \hline
$\sigma$ (Real Data) & \multicolumn{3}{c}{908.70}           &  & \multicolumn{3}{c}{1293.78}           &  &  \\
RMSE          & 727.31      & 1083.22        & 834.32 &  & 967.46      & 740.73         & 740.73 &  &  \\
MASE          & 0.1922      & 0.3152         & 0.2348 &  & 0.1312      & 0.0947         & 0.0947 &  & 
\end{tabular}

\vspace{0.5cm}

\begin{tabular}{c|ccccccccc}
              & 8h baseline & 8h allow drift & 8h     &  & \color{red}{12h baseline} & 12h
              allow drift & 12h     &  &  \\ \hline
$\sigma$ (Real Data) & \multicolumn{3}{c}{1691.32}           &  & \multicolumn{3}{c}{2154.85}           &  &  \\
RMSE          & 1205.49      & 999.47        & 999.47 &  & \color{red}{1565.43}    & 1722.83        & 1722.83 &  &  \\
MASE          & 0.0983      & 0.0905         & 0.0905 &  & \color{red}{0.0857}     & 0.0953         & 0.0953 &  & 
\end{tabular}

\vspace{0.5cm}

\begin{tabular}{c|ccc}
              & 24h baseline & 24h allow drift & 24h  \\ \hline
$\sigma$ (Real Data) & \multicolumn{3}{c}{1752.65}     \\
RMSE          & 2329.76     & 1646.92         & 1646.92   \\
MASE          & 0.4198      & 0.2925         & 0.2925 \\
\end{tabular}

\vspace{0.5cm}

\caption{Case 2: Forecast errors for different interval sizes (best result in
red)}\label{tab:case2_interval}
\end{table}


\begin{table}[!ht]
\footnotesize
\begin{tabular}{c|ccccccccc}
              & 4h baseline & 4h allow drift & 4h     &  & 6h baseline & 6h allow drift & 6h     &  &  \\ \hline
$\sigma$ (Real Data) & \multicolumn{3}{c}{929.73}           &  & \multicolumn{3}{c}{1334.01}           &  &  \\
RMSE          & 737.22      & 692.61         & 692.61 &  & 985.86      & 930.13        & 930.13 &  &  \\
MASE          & 0.1548      & 0.1441         & 0.1441 &  & 0.1071      & 0.0998        & 0.0998 &  & 
\end{tabular}

\vspace{0.5cm}

\begin{tabular}{c|ccccccccc}
              & 8h baseline & 8h allow drift & 8h     &  & 12h baseline & \color{red}{12h allow drift} & 12h     &  &  \\ \hline
$\sigma$ (Real Data) & \multicolumn{3}{c}{1738.75}           &  & \multicolumn{3}{c}{2203.12}           &  &  \\
RMSE          & 1197.45     & 1165.13        & 1165.13  &  & 1581.66     & \color{red}{1453.84}      & 1453.84 &  &  \\
MASE          & 0.0778      & 0.0763         & 0.0763  &  & 0.0695      & \color{red}{0.0595}        & 0.0595 &  & 
\end{tabular}

\vspace{0.5cm}

\begin{tabular}{c|ccc}
              & 24h baseline & 24h allow drift & 24h  \\ \hline
$\sigma$ (Real Data) & \multicolumn{3}{c}{1686.92}     \\
RMSE          & 2275.91     & 2547.73         & 2260.99   \\
MASE          & 0.3304      & 0.3525         & 0.3044 \\
\end{tabular}

\vspace{0.5cm}

\caption{Case 3: Forecast errors for different interval sizes (best result in
red)}\label{tab:case3_interval}
\end{table}


\section{Segmentation}

The experiments described on this section use all the phases of the approach.
This experiments will use datasets generated on artificial environment in order to
test the ability of the approach to capture changes on the volumes of some
characteristics of the dataset.
