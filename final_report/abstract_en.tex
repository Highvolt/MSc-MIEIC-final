\chapter*{Abstract}

The online advertisement industry handles a large quantity of money and users
everyday.
This industry is always trying to get more efficient, for example, by enhancing
the targeting of online advertising campaigns. 

This pursuit of efficiency on the world of online advertising turned simpler
methods of prediction unable to report an accurate number of
impressions, used to calculate the value
of a publisher's inventory. The introduction of concepts like frequency capping made that very clear.

There is now the need not only to predict the number of visits, but also to
predict when this visits will happen, what the user did before going to that
website and who he is.

In this document that concept will be approached using Data Mining techniques,
such as clustering and time series analyses, in order to generate a future ad request log
using only past data.

This generated results will be in the same format as the input dataset, to be used on simulators
capable of calculate important metrics, for publishers and advertisers, for a set
of campaigns.

\chapter*{Resumo}

O mercado da publicidade online envolve diariamente muito dinheiro e muitos utilizadores. 
Este mercado que esta constantemente à procura de formas de se tornar
mais eficiente, por exemplo, melhorando a selecção do público alvo das suas campanhas
publicitárias. 

Esta busca pela eficiencia no mercado da publicidade online tornou metodos de
previsão mais simples incapazes de calcular correctamente o número de
impressões, utilizadas para calcular o valor do inventario de um
\textit{publisher}.
A introdução de conceitos como o \textit{frequency capping} torna isso muito evidente.

Há actualmente a necessidade de não só prever o número de visitas, como também
quando vão ocorrer essas visitas, o que o utilizador fez antes de lá chegar e
quem é
o utilizador em questão.

Neste trabalho esse conceito vai ser abordado recorrendo a técnicas de
\textit{Data Mining}, como o \textit{clustering} e analise de series temporais, de forma a conseguir gerar
um registo futuro de pedidos de publicidade utilizando apenas dados passados.

Os registos gerados estarão prontos a serem posteriormente utilizados, em
simuladores capazes de calcular os resultados, para um universo de campanhas.
