\chapter{Conclusions and Future Work} \label{chap:concl}

\section*{}

\section{Objectives Fulfilment}

We proposed the use of the three phase approach (Chapter~\ref{chap:chap4}) in
order to obtain a richer future prediction of web activity on a network.
The first phase allows better predictions for certain characteristics,
the second phase allows the prediction of volumes in the future, and lastly the third
phase fills the predicted future data with additional information from the past
while following
the tends of the predicted volumes.

Our results show that the usage of dataset segmentation returns good to moderate
improvements while predicting the volumes.

At the end we get a dataset with the same format as the one given as input, but with future
data that follows the main characteristics of the past while amplifying some of the
more important characteristics for the future.

This dataset is completely ready to be used on simulations of online advertising
campaigns in order to perceive their behaviour in the future.

\section{Future Work}

In order to achieve better results, some other approaches could be explored. For
example, we could use $MLP$, multilayer perceptron to predict the time series values
instead of the $ARIMA$.

It would also make sense to explore the usage of additional time series for new
entries of each parameter, for example in order to correctly predict the occurrence of a new
website. This phase will also need additional knowledge about
some of the parameters to get better results (for example know how to generate a
new domain). This knowledge should be optional in order to maintain the data
agnostic approach, but when available it would probably would improve the results.

Another road to be explored is segmentation phase for example the usage
recursive parameter segmentation. For example, segment the whole dataset by
browser and then by country. It is my believe that this approach would get good
results if a huge volume of data were available.

It would also be interesting to test the resultant datasets on a real online
advertising campaign simulator in order to better assess the results in practical
applications.

