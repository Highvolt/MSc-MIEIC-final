\chapter{Conclusions and Future Work} \label{chap:concl}

\section*{}

\section{Objectives Fulfilment}


\section{Future Work}

In order, to achieve better results some other approaches could be explored. For
example, use $MLP$, multilayer perception to predict the time series values
instead of the $ARIMA$.

It would also make sense to explore the usage of additional time series for new
entries of each variable in order to correctly predict the occurrence of a new
website, for example. This phase will also need an additional knowledge about
some of the parameters to get better results, for example know how to generate a
new domain. This knowledge should be optional in order to maintain the data
agnostic approach, but when available probably would improve the results.

Another road to be explored is segmentation phase for example the usage
recursive parameter segmentation. For example, segment the whole dataset by
browser and then by country. It is my believe that this approach would get good
results if a huge volume of data were available.

